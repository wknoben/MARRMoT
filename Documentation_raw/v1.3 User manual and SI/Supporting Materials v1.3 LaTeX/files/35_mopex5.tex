\subsection{MOPEX-5 (model ID: 35)}
The MOPEX-5 model (fig.~\ref{fig:35_schematic}) is part of a model improvement study that investigates the relationship between dominant processes and model structures for 197 catchments in the MOPEX database \citep{Ye2012}. It has 5 stores and 12 parameters ($T_{crit}$, $ddf$, $S_{b1}$, $t_w$, $I_{\alpha}$, $I_{s}$, $T_{min}$, $T_{max}$, $S_{b2}$, $t_u$, $S_e$ and $t_c$). The original model relies on observations of Leaf Area Index and a calibrated interception fraction. \citet{Liang1994} show typical Leaf Area Index time series, and a sinusoidal function is a reasonable approximation of this. Therefore, the model is slightly modified to use a calibrated sinusoidal function, so that the data input requirements for MOPEX-5 are consistent with other models.  The model aims to represent:

\begin{itemizecompact}
\item Snow accumulation and melt;
\item Time-varying interception and the impact of phenology on transpiration;
\item Saturation excess flow;
\item Infiltration to deeper soil layers;
\item A split between fast and slow runoff.
\end{itemizecompact}

\subsubsection{File names}
\begin{tabular}{@{}ll}
Model: &m\_35\_mopex5\_12p\_5s \\
Parameter ranges: &m\_35\_mopex5\_12p\_5s\_parameter\_ ranges \\
\end{tabular}

% Equations
\subsubsection{Model equations}

% Model layout figure
{ 																	% This ensures it doesn't warp text further down
\begin{wrapfigure}{l}{6cm}
\includegraphics[trim=1cm 13cm 7cm 1cm,width=7cm,keepaspectratio]{./files/35_schematic.pdf}
\caption{Structure of the MOPEX-5 model} \label{fig:35_schematic}
\end{wrapfigure}

\begin{align}
	\frac{dS_n}{dt} &= P_s-Q_{n} \\
	P_s &= \begin{cases}
		P, &\text{if } T \leq T_{crit} \\
		0, & \text{otherwise} \\
	\end{cases} \\
	Q_n &=\begin{cases}
		ddf*(T-T_{crit}), &\text{if } T > T_{crit} \\
		0, & \text{otherwise} \\
	\end{cases}
\end{align}

Where $S_n$ [mm] is the current snow pack. Precipitation occurs as snowfall $P_s$ $[mm/d]$ when current temperature T $[^oC]$ is below threshold $T_{crit}$ $[^oC]$. Snowmelt $Q_N$ $[mm/d]$ occurs when the temperature rises above the threshold temperature and relies in the degree-day factor $dd$ $[mm/^oC/d]$.

} % end of wrapfigure fix
\vspace{5cm}

\begin{align}
	\frac{dS_1}{dt} &= P_r-ET_1-I-Q_{1f}-Q_w \\
	P_r &= \begin{cases}
		P, &\text{if } T > T_{crit} \\
		0, & \text{otherwise} \\
	\end{cases} \\
	ET_{c1} &= \frac{S_1}{S_{b1}}*Ep_c\\
	I &= max\left(0,I_{\alpha} + (1-I_{\alpha})sin\left(2\pi\frac{t+I_{s}}{365/d}\right)\right)\\
	Q_{1f} &= \begin{cases}
		P, &\text{if } S_1 \geq S_{b1} \\
		0, & \text{otherwise} \\
	\end{cases} \\
	Q_w &= t_w*S_1
\end{align}


Where $S_1$ [mm] is the current storage in soil moisture and $P_r$ precipitation as rain $[mm/d]$. Evaporation $ET_1$ $[mm/d]$ depends linearly on current soil moisture, maximum soil moisture $S_{b1}$ [mm] and phenology-corrected potential evapotranspiration: 

\begin{align}
	Ep_c &= Ep*GSI \\
	GSI &= \begin{cases}
		0 , &\text{if } T < T_{min} \\
		\frac{T-T_{min}}{T_{max}-T_{min}}, &\text{if } T_{min} \geq T < T_{max} \\
		1, &\text{if } T \geq T_{max} \\
	\end{cases}
\end{align}

Where GSI is a growing season index based on parameters $T_{min}$ $[^oC]$ and $T_{max}$ $[^oC]$. Interception $I$ $[mm/d]$ depends on the mean intercepted fraction $I_{\alpha}$ [-] and the maximum Leaf Area Index timing $I_{s}$ [d]. Saturation excess flow $Q_{1f}$  $[mm/d]$ occurs when the soil moisture bucket exceeds its maximum capacity. Infiltration to deeper groundwater $Q_w$  $[mm/d]$ depends on current soil moisture and time parameter $t_w$  $[d^{-1}]$.

\begin{align}
	\frac{dS_2}{dt} &= Q_w-ET_2-Q_{2u} - Q_{2f}\\
	ET_{c2} &= \frac{S_2}{S_{e}}*Ep_c\\
	Q_{2u} &= t_u*S_2\\
	Q_{2f} &= \begin{cases}
		Q_w, &\text{if } S_2 \geq S_{b2} \\
		0, & \text{otherwise} \\
	\end{cases}
\end{align}

Where $S_2$ [mm] is the current groundwater storage, refilled by infiltration from $S_1$. Evaporation $ET_2$ $[mm/d]$ depends linearly on current groundwater and root zone storage capacity $S_e$ [mm]. Leakage to the slow runoff store $Q_{2u}$ $[mm/d]$ depends on current groundwater level and time parameter $t_u$ $[d^{-1}]$. When the store reaches maximum capacity $S_{b2}$ [mm], excess flow $Q_{2f}$ $[mm/d]$ is routed towards the fast response routing store.

\begin{align}
	\frac{dS_{c1}}{dt} &= Q_{1f}+Q_{2f}-Q_{f}\\
	Q_f &= t_c*S_{c1}
\end{align}

Where $S_{c1}$ [mm] is current storage in the fast flow routing reservoir, refilled by $Q_{1f}$ and $Q_{2f}$. Routed flow $Q_f$ depends on the mean residence time parameter $t_c$ $[d^{-1}]$.

\begin{align}
	\frac{dS_{c2}}{dt} &= Q_{2u}-Q_{u}\\
	Q_u &= t_c*S_{c2}
\end{align}

Where $S_{c2}$ [mm] is current storage in the slow flow routing reservoir, refilled by $Q_{2u}$. Routed flow $Q_u$ depends on the mean residence time parameter $t_c$ $[d^{-1}]$. Total simulated flow $Q_t$ $[mm/d]$:

\begin{align}
	Q_t &= Q_f + Q_u
\end{align}


\subsubsection{Parameter overview}
% Table generated by Excel2LaTeX from sheet 'Sheet1'
\begin{table}[htbp]
  \centering
    \begin{tabular}{lll}
    \toprule
    Parameter & Unit  & Description \\
    \midrule
    $T_{crit}$ & $^oC$ & Threshold temperature for snowfall and melt \\
    $ddf$ & $mm~^oC^{-1}~d^{-1}$ & Degree-day factor \\
    $S_{b1}$ & $mm$  & Maximum soil moisture storage \\
    $t_w$ & $d^{-1}$ & Runoff coefficient \\
    $I_{\alpha}$ & $-$   & Mean intercepted fraction of precipitation \\
    $I_{s}$ & $d$   & Timing of peak interception capacity \\
    $T_{min}$ & $^oC$ & Minimum temperature to start growing season \\
    $T_{max}$ & $^oC$ & Temperature of maximum plant growth \\
    $S_{b2}$ & $mm$  & Maximum deep storage \\
    $t_u$ & $d^{-1}$ & Runoff coefficient \\
    $S_e$ & $mm$  & Maximum groundwater storage capacity \\
    $t_c$ & $d^{-1}$ & Runoff coefficient \\
    \bottomrule
    \end{tabular}%
  \label{tab:addlabel}%
\end{table}%

